\addchap{A Ergänzungen}
\setcounter{chapter}{1}

\section{Details zu bestimmten theoretischen Grundlagen}

\section{Weitere Details, welche im Hauptteil den Lesefluss behindern}

\addchap{B Details zu Laboraufbauten und Messergebnissen}
\setcounter{chapter}{2}
\setcounter{section}{0}
\setcounter{table}{0}
\setcounter{figure}{0}

\section{Versuchsanordnung}

\section{Liste der verwendeten Messgeräte}

\section{Übersicht der Messergebnisse}

\section{Schaltplan und Bild der Prototypenplatine}

\addchap{C Zusatzinformationen zu verwendeter Software}
\setcounter{chapter}{3}
\setcounter{section}{0}
\setcounter{table}{0}
\setcounter{figure}{0}

\section{Struktogramm des Programmentwurfs}

\section{Wichtige Teile des Quellcodes}

\addchap{D Datenblätter}
\setcounter{chapter}{4}
\setcounter{section}{0}
\setcounter{table}{0}
\setcounter{figure}{0}

%\section{Einbinden von PDF-Seiten aus anderen Dokumenten}

Auf den folgenden Seiten wird eine Möglichkeit gezeigt, wie aus einem anderen PDF-Dokument komplette Seiten übernommen werden können, z.~B. zum Einbindungen von Datenblättern. Der Nachteil dieser Methode besteht darin, dass sämtliche Formateinstellungen (Kopfzeilen, Seitenzahlen, Ränder, etc.) auf diesen Seiten nicht angezeigt werden. Die Methode wird deshalb eher selten gewählt. Immerhin sorgt das Package \textit{\glqq pdfpages\grqq}~für eine korrekte Seitenzahleinstellung auf den im Anschluss folgenden \glqq nativen\grqq~\LaTeX-Seiten.

Eine bessere Alternative ist, einzelne Seiten mit \textit{\glqq$\backslash$includegraphics\grqq}~einzubinden.

\includepdf[pages={2-4}]{docs/EingebundenesPDF.pdf}

\clearpage
