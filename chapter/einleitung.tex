\chapter{Einleitung}
\label{cha:Einleitung}

Folgende Stichworte können zum Aufbau der Einleitung herangezogen werden.

\begin{itemize}
\item Hinführung, Begründung, Zweck und Ziel der Aufgabenstellung
\item Erläuterung der Problemstellung
\item Konkretisierung der zu lösenden Aufgabe
\item Gegebenenfalls Formulierung einer Leitfrage oder Forschungsfrage
\item Ausgangslage, geplante Vorgehensweise, Methoden zur Bearbeitung und Zielsituation
\item Zum Ende der Einleitung wird eine Kurzübersicht über die Inhalte der Kapitel gegeben: \glqq Die Arbeit ist wie folgt gegliedert: ...\grqq
\end{itemize}

Die Einleitung wird üblicherweise auf ein bis zwei Seiten als fortlaufender Text geschrieben. Eine weitere Untergliederung in nummerierte Abschnitte ist nicht empfehlenswert, da dies erstens unüblich ist, zweitens die Lesbarkeit nicht begünstigt und drittens die Formulierung der Einleitung erschwert. Weitere Empfehlungen zum Aufbau der Einleitung und des gesamten Dokuments sind z.~B. aus \autocite{DHBW.2021} und \autocite{Lindenlauf.2022} zu entnehmen.

\clearpage

Hinweise: 

\begin{itemize}
\item Auch in der Einleitung unbedingt zu wichtigen Hintergründen und Fakten Zitate aufführen.\index{Zitat} Zitate bitte in der Form \autocite{Tipler.2019} oder mit Seitenbezug \autocite[66]{Ziegler.2017} oder auch mehrere Zitate  \autocite{Tipler.2019, Ziegler.2017} innerhalb einer eckigen Klammer angeben. Zur besseren Lesbarkeit bitte immer ein Leerzeichen vor dem Zitat einfügen.
	
\item Bereits in der Einleitung können Abkürzungen erläutert werden. Grundsätzlich gilt, dass bei der ersten Verwendung einer Abkürzung diese auch erläutert wird. Zum Beispiel können das \acrlong{acr:ABS}  (\acrshort{acr:ABS}) oder die Fahrdynamikregelung (\acrlong{acr:ESC}, \acrshort{acr:ESC}) als Abkürzungen eingeführt werden. In der Datei \textit{pages/abkuerzungen.tex} sind alle verwendeten Abkürzungen einzufügen. Neben dem verpflichtenden Abkürzungsverzeichnis kann auch ein Glossar hinzugefügt werden. In dieser Vorlage können Glossareinträge in der Datei \textit{pages/glossar.tex} eingefügt werden. Ein \Gls{gls:glossar} ist jedoch nicht verpflichtend.
\end{itemize}